% ************************** Thesis Abstract *****************************
% Use `abstract' as an option in the document class to print only the titlepage and the abstract.
\begin{abstract}
    Cloud providers detain an oligarchical dominance on the market of computational power. Recent developments in blockchain technologies and decentralized markets can potentially solve this issue. We explored the possibility of creating a decentralized market for computational power by using the Ethereum blockchain and Docker containers.
    In order to do this, we designed a protocol to exchange computations requests and rewards for executing them. Additionally, we developed a smart contract implementing this protocol and a client application that allows users to interact with this contract.

    \vspace{1cm}

    I cloud provider detengono un dominio oligarchico del mercato del potere computazionale. I recenti sviluppi nelle tecnologie blockchain e nei mercati decentralizzati potrebbero concretizzarsi in una soluzione per questo problema. Abbiamo esplorato la possibilità di creare un mercato decentralizzato per il potere computazionale utilizzando la blockchain Ethereum e i containers Docker.
    Per poter fare questo, abbiamo inventato un protocollo per scambiare richieste di esecuzione di computazioni e ricompense per eseguirle. In aggiunta, abbiamo sviluppato uno smart contract che implementa questo protocollo ed un'applicazione che permette agli utenti di interagire con questo contratto.

    \vspace{1cm}

    Cloud-Anbieter besitzen eine oligarchische Herrschaft über den Markt der Rechenleistung. Jüngste Entwicklungen in Blockchain-Technologien und die Dezentralisierung der Märkte können dieses Problem möglicherweise lösen. Wir haben die Möglichkeit untersucht, einen dezentralen Markt für Rechenleistung mit der Ethereum Blockchain und Docker-Containern zu schaffen.
    Um dies zu erreichen, haben wir ein Protokoll erfunden, um Anfragen für die Ausführung von Berechnungen, sowie Belohnungen, auszutauschen. Darüber hinaus haben wir einen Smart Contract entwickelt, der dieses Protokoll implementiert, und eine Clientanwendung, die es Benutzern ermöglicht, mit diesem Vertrag zu interagieren.

\end{abstract}
